\documentclass[final]{union-cs-thesis}
% Use \documentclass[condensed]{union-cs-thesis} to cause the layout to
% take fewer pages.  It uses single-spacing instead of double spacing and
% compresses the front matter. This might be useful as you draft your
% thesis.

% Use \documentclass[final]{union-cs-thesis} for the final version.

% percent sign starts a comment, goes to end of line

\usepackage{graphicx}
\usepackage[noend]{algorithmic}


\begin{document}

\title{Computer Science Senior Thesis \LaTeX{} template}
\author{Aaron G. Cass\footnote{Based on an earlier version by John Rieffel}}
\maketitle

\begin{abstract}

This document provides the basic format and style recommended for a
Computer Science Senior Design project at Union College.  After a
brief introduction we illustrate the various sections we recommend, and
provide brief explanations of their contents.  Parenthetical comments
provide guidance for content.  Many further tips are included in the
latex source. (Your abstract should provide a concise
overview of your thesis: a brief introduction of the problem, a
summary of the background, a description of your research question, a
description of your research design and implementation, a summary of
your results and conclusion. Often, people only read the abstract (and
captions) of papers!)

\end{abstract}

\tableofcontents %that's all you need!
\listoffigures %again easy
\listoftables %as pie

\section{Introduction} % \section* would create section without section
                       % number.

The Computer Science Thesis is the culmination of your CS
education at Union College.  By the end of this capstone experience,
typically winter of your senior year, you will have created and
implemented your own full-fledged independent research project.  The
process begins with the CSC497 seminar, in which you continue to
develop the skills necessary for independent research to which you were
introduced in your Sophomore Research Seminar (SRS).  The Senior
thesis serves as a way to document and present your capstone project. 

Typically, each term of your capstone experience includes an intensive
writing experience, resulting in a substantive final document.  This
LaTeX template should serve for all three written projects, with only
modest changes to the template itself.  Descriptions of each term's
outcomes are as follows:

\begin{itemize}
\item{CSC 497} A well crafted and fully formed research proposal,
describing your research question and design methodology.  This
proposal will also include a comprehensive study of prior work along
with an annotated bibliography.

\item{CSC 498} A well crafted and finely detailed research design
document.  This will include elaborations on your paper from 497 (a
well crafted, motivated and  contextualized research question, and a
comprehensive study of prior work), as well as improvements to your
research methods and plans for data acquisition and analysis.

\item{CSC 499} Your Senior Thesis.  This will include elaborations on your
paper from 498, as well as description and analysis of the data
gathered during your research.  Your advisor may also ask you to write
a shorter 8-page paper worthy of submission to an academic conference.

\end{itemize}

We have written this document as a means to formalize and normalize
written reports across projects and advisors.  {\em This should only
serve as a guide and template: no matter what, be sure
to consult with your own project advisor to confirm their specific
requirements for your thesis!}  After all, they will determine your
final grade.

(In this section you need to motivate your entire research project.
What did you work on and why is it important?)

% notice I've given this section a label with the \label command
\section{Background and Related Work}\label{sec:background}

This can be a section where you contextualize research that others
have done.  Here are some random citations to get you started.  See
the .bib file for details.  Trimmer~\cite{trimmer-goqbot}wrote a neat paper about soft
robots.  Clune {\em et al}~\cite{clune-endless} have done some neat
work with evolving objects.  Many other people have done cool
research~\cite{bongard-3d-cppn,Lobo:form}.  Dawkins even wrote a book
about it~\cite{dawkins:blind}.  Occassionally you'll need to cite a
web page like the CS Dept's page~\cite{cswebpage}.
\begin{figure}
\centering
\includegraphics[width=2in]{CS_LOGO.pdf}
\caption{This is the test image caption.  Generally images are first
saved in PDF or EPS format.  You can use other LaTeX packages (like
graphicx) for other image
formats.} \label{fig:cslogo}
\end{figure}

\section{Methods and Design} \label{sec:methods}

(You might want a section like this, but talk to your advisor about
the exact format they request).

This guide is written in the LaTeX markup language, and edited in
emacs (you may use the editor of your choice, of course) and compiled
using pdflatex.  A Makefile is included in the directory, and is a
useful way to compile edits.

LaTeX is fantastic for many reasons, but one of my favorite is the
ability to use labels to refer to figures, sections, formulae, etc.
For instance I may want to refer to Section~\ref{sec:background} in
the text here, or maybe to Figure~\ref{fig:cslogo}.  As you add and remove sections and subsections, LaTeX
intelligently renumbers references accordingly.

See the source code:
Often you may want to include psuedocode in your thesis.  The {\em
algorithmic} package can help you with that.  Here is psuedocode for
compiling your thesis:

%this is using the algorithmic package
\begin{algorithmic}[1]
\WHILE{you see compile errors or need to make changes}
  \STATE edit source in text editor of choice
  \STATE compile by typing ``make''
\ENDWHILE
\end{algorithmic}

Instead, you may want verbatim code in your thesis.  Use this
sparingly!
%note spaces work as whitespace, tabs do not
\begin{verbatim}
define factorial (n):
     if (n == 0):
        return 1
     else:
        return n*factorial(n-1)
\end{verbatim}

(Your methods and design section should describe everything someone
might need to replicate your experiment, along with explanation of
design choices along the way).

\section{Another Section}

Often you will perform experiments or human subject studies as part of
your thesis research.  Here you might describe the {\em actual} experiments
performed, as opposed to the {\em experimental setup} described in
section~\ref{sec:methods} above.  Or you may need another section (or
sections) entirely).

\subsection{a subsection}

Look! we can make subsections!

\subsubsection{a subsubsection}

and sub sub sections!

\paragraph{there are no subsubsubsections}

You may use a paragraph construct instead.

\section{Yet Another Section}

This might be a good place for an example table.  Table \ref{tab:timeline}
shows a time line for a CS Thesis.  Note that we are writing in {\bf all}
of the terms --- we are not waiting until the end of the CSC499 to write up
the thesis.  Note: if you just use a tabular environment outside of a table
environment, you still get a table, but it will be placed directly in the
text where you put it, and it will not be shown in the List of Tables.

\begin{table}
\begin{center}
\begin{tabular}{r c l}
{\bf Term} &{\bf Course} &{\bf Aims/Outcomes} \\ \hline
Junior Spring & CSC497 & Find Advisor, develop ideas\\ 
&&{\bf A research question and fully formed proposal}\\ \hline
Senior Fall& CSC498 & Do lots of work and writing\\
Senior Winter & CSC499 & More work, more writing\\
&&{\bf A Completed Capstone Design
Project}\\ \hline
\end{tabular}
\end{center}
\caption{A time line for a CS Thesis\label{tab:timeline}}

\end{table}


\section{Final Section}

All good things must come to an end.

\bibliography{thesis}  
\bibliographystyle{plain}
\end{document}
